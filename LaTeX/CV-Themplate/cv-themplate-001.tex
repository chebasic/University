% resume.tex
% vim:set ft=tex spell:

\documentclass[10pt,letterpaper]{article}
\usepackage[letterpaper,margin=0.75in]{geometry}
\usepackage[utf8]{inputenc}
\usepackage{mdwlist}
\usepackage[T1]{fontenc}
\usepackage{textcomp}
\usepackage{tgpagella}
\pagestyle{empty}
\setlength{\tabcolsep}{0em}

% indentsection style, used for sections that aren't already in lists
% that need indentation to the level of all text in the document
\newenvironment{indentsection}[1]%
{\begin{list}{}%
	{\setlength{\leftmargin}{#1}}%
	\item[]%
}
{\end{list}}

% opposite of above; bump a section back toward the left margin
\newenvironment{unindentsection}[1]%
{\begin{list}{}%
	{\setlength{\leftmargin}{-0.5#1}}%
	\item[]%
}
{\end{list}}

% format two pieces of text, one left aligned and one right aligned
\newcommand{\headerrow}[2]
{\begin{tabular*}{\linewidth}{l@{\extracolsep{\fill}}r}
	#1 &
	#2 \\
\end{tabular*}}

% make "C++" look pretty when used in text by touching up the plus signs
\newcommand{\CPP}
{C\nolinebreak[4]\hspace{-.05em}\raisebox{.22ex}{\footnotesize\bf ++}}

% and the actual content starts here
\begin{document}

\begin{center}
{\Huge \textrm{Carlos José García Santos}}

\vspace*{0.25cm}
\ \ Quijorna, Madrid 28693
\\
(+34) 619262412\ \textbullet
\ \ cj.garciasan@gmail.com
\end{center}

\hrule
\vspace{1cm}

\vspace{-0.4em}
\subsection*{Perfil}

\begin{itemize}
	\parskip=0.1em

	\item
	\headerrow
		{\textbf{Personal}}
		{\textbf{}}
	\\
	\headerrow
		{\emph{}}
		{\emph{}}
	\begin{itemize*}
		\item Estudiante de Ingeniería Informática por la Universidad Politécnica de Madrid,\\ motivado y trabajador.
		Actualmente buscando experiencia y la oportunidad de\\ adquirir conocimientos más allá de lo teórico y las aulas.
	\end{itemize*}
	

\vspace{1.4cm}
\hrule
\vspace{-0.4em}
\subsection*{Educación}

\begin{itemize}
	\parskip=0.1em

	\item 
	\headerrow
		{\textbf{Universidad Politécnica de Madrid}}
		{\textbf{Boadilla, Madrid}}
	\\
	\headerrow
		{\emph{Escuela Técnica Superior de Ingenieros Informáticos}}
		{\emph{Actualidad}}
	\begin{itemize*}
		\item Grado en Ingeniería Informática.
	\end{itemize*}
	
	\item 
	\headerrow
	{\textbf{IES Las Encinas}}
	{\textbf{Villanueva de la Cañada, Madrid}}
	\\
	\headerrow
	{\emph{}}
	{\emph{2005 -- 2011}}
	\begin{itemize*}
		\item Educación Secundaria Obligatoria.
		\item Educación Secundaria, Bachillerato.
	\end{itemize*}
	

\end{itemize}



\vspace{-0.4em}
\subsection*{Conocimientos Informáticos.}

\begin{indentsection}{\parindent}
\hyphenpenalty=1000
\begin{description*}
	 \textbf{Lenguajes Y Tecnologías :}	
		\subitem Básico: Python, Ruby, Ruby on Rails, JavaScript.
		\subitem Medio: HTML/CSS, \CPP, C\#, PHP, \LaTeX.
		\subitem Avanzado: C, Java,\LaTeX, Bash, SQL.\\
	 \textbf{Sistemas Operativos :}			
		 \subitem Microsoft Windows.
		 \subitem Linux.
		

\end{description*}
\end{indentsection}

\subsection*{Idiomas.}

\begin{indentsection}{\parindent}
	\hyphenpenalty=1000
	\begin{description*}
		
		\subitem \textbf{Español:} Nativo
		\subitem \textbf{Inglés :} Nivel B2.
		
	\end{description*}
\end{indentsection}

\end{itemize}
\end{document}
